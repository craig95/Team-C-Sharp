\documentclass[a4paper,12pt]{article}
\usepackage{array}
\newcolumntype{L}{>{\centering\arraybackslash}m{8cm}}

%	Mia Gerber 
%	15016502
%	
%	COS 301: Assignment 2
%
%	Navigation Subsystem: Requirements, Constraints and Attributes

\begin{document}
	\newpage
	\section{Navigation}
	
	\subsection{External Interface Requirements}
	\paragraph 
		These requirements relate to both interfacing with other subsystems within the UP Nav system as well as interfacing 
		with the hardware that the application will be deployed on.
	\begin{itemize}
		\small
		\item To "Notifications" module will be used to give directions to the user during navigation in the form of push notifications.
		\item The "Points of interest" module will be seen by the Navigation module simply as destinations to be navigated to or as a current location.
		\item 
	\end{itemize}
	
	\subsection{Performance Requirements}
	\begin{itemize}
		\small
		\item 	
	\end{itemize}
	
	\subsection{Design Constraints}
	\begin{itemize}
		\small
		\item The use of design patterns is crucial to object oriented software engineering but we are limited 
			in the types of design patterns we are able to use within the subsystems due to the requirements 
			imposed upon the system as a whole.
		\item The Navigation subsystem receives critical information from the GIS subsystem to ensure that the route
			calculated is correct and to recalculate the route in real time if the user goes off track.The coupling between
			these two modules will have to be high to ensure the user's needs are met.\
	\end{itemize}
	
	\subsection{Software System Attributes}
	\begin{itemize}
		\small 
		\item High cohesion with low coupling allowing for easy addition, removal or replacement of modules.
		\item Polymorphism will be implemented as means of specifying object behaviour when the user imposes restrictions
			upon the base behaviour of the object.
	\end{itemize}
	
	\subsection{Design Patterns used}
	\paragraph{Discussion of design patterns implemented in UML diagrams:}
		Memento: To save a route for future use
		State: Keep track of user's progress along the route. Have a state for each waypoint
		Strategy: Provide different means of navigating the same route
	
\end{document}
